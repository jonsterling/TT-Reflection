\documentclass{amsart}

\usepackage{proof}

\title{A Type Theory with Scoped Equality Reflection}
\author{Jonathan Sterling}


\begin{document}
\maketitle

\def\emp{\diamond}
\def\sig{\mbox{ sig}}
\def\ctx{\mbox{ ctx}}
\def\hints{\mbox{ hints}}
\def\fresh{\mathrel{\#}}
\def\infers{\Rightarrow}
\def\checks{\Leftarrow}
\def\uni{\mathbb{U}}

Feel free to change the stuff below as our thinking evolves. This is just a
starting point. We have the following judgments:

\bigskip
\begin{tabular}{ll}
$\vdash\Sigma\sig$                            & $\Sigma$ is a valid signature\\
$\mathcal{H}\vdash_\Sigma\Gamma\ctx$          & $\Gamma$ is a valid context\\
$\Gamma\vdash_\Sigma\mathcal{H}\hints$        & $\mathcal{H}$ is a valid hints set\\
$\Gamma;\mathcal{H}\vdash_\Sigma M\infers A$ & $M$ infers type $A$\\
$\Gamma;\mathcal{H}\vdash_\Sigma M\checks A$  & $M$ checks type $A$\\
$\Gamma;\mathcal{H}\vdash_\Sigma M\equiv N:A$ & $M$ equals $N$ at type $A$
\end{tabular}
\bigskip

\noindent
The rules for the theory are given inductive-recursively below.

\begin{equation}
  \infer{\vdash\emp\sig}{}
  \qquad
  \infer{
    \vdash\Sigma,c:A\sig
  }{
    \begin{array}{l}
      \vdash\Sigma\sig\\
      \emp\vdash_\Sigma^\emp A\checks\uni
    \end{array} &
    \Sigma\fresh c
  }
  \tag{Signatures}
\end{equation}

\begin{equation}
  \infer{\mathcal{H}\vdash_\Sigma\emp\ctx}{}
  \qquad
  \infer{
    \mathcal{H}\vdash_\Sigma\Gamma,x:A\ctx
  }{
    \begin{array}{l}
      \mathcal{H}\vdash_\Sigma\Gamma\ctx\\
      \Gamma;\mathcal{H}\vdash_\Sigma A\checks\uni\\
    \end{array} &
    \Gamma\fresh x
  }
  \tag{Contexts}
\end{equation}

\begin{equation}
  \infer{\Gamma\vdash \emp\hints}{}
  \qquad
  \infer{
    \Gamma\vdash_\Sigma \mathcal{H}, a\equiv b:A\hints
  }{
    \begin{array}{l}
      \Gamma\vdash_\Sigma \mathcal{H}\hints\\
      \Gamma;\mathcal{H}\vdash_\Sigma A\checks\uni\\
      \Gamma;\mathcal{H}\vdash_\Sigma a,b\checks A
    \end{array}
  }
  \tag{Hints}
\end{equation}

\begin{equation}
  \infer{
    \Gamma;\mathcal{H}\vdash_\Sigma x\infers A
  }{
    x:A\in\Gamma
  }
  \qquad
  \infer{
    \Gamma;\mathcal{H}\vdash_\Sigma c\infers A
  }{
    c:A\in\Sigma
  }
  \qquad
  \infer{
    \Gamma;\mathcal{H}\vdash_\Sigma a\equiv b:A
  }{
    a\equiv b:A \in \mathcal{H}
  }
  \qquad
  \tag{Projection}
\end{equation}

\begin{equation}
  \infer{
    \Gamma;\mathcal{H}\vdash_\Sigma \mathsf{reflect}\ p\ \mathsf{in}\ e \checks C
  }{
    \begin{array}{l}
      \Gamma;\mathcal{H}\vdash_\Sigma p \infers \mathsf{Id}_A\,a\;b\\
      \Gamma;\mathcal{H}\vdash_\Sigma C\checks\uni\\
      \Gamma;(\mathcal{H}, a\equiv b:A)\vdash_\Sigma e \checks C
    \end{array}
  }
  \tag{Reflect}
\end{equation}

\begin{equation}
  \infer{
    \Gamma;\mathcal{H}\vdash_\Sigma\mathrm{Q}_A[x]B\infers\uni
  }{
    \begin{array}{l}
      \Gamma;\mathcal{H}\vdash_\Sigma A \checks\uni\\
      \Gamma,x:A;\mathcal{H}\vdash_\Sigma [x]B\checks\uni
    \end{array} &
    \mathrm{Q} \in \{ \Sigma, \Pi \}
  }
  \tag{Binders}
\end{equation}

\begin{equation}
  \infer{
    \Gamma;\mathcal{H}\vdash_\Sigma \langle a,b\rangle \checks \Sigma_A[x]B
  }{
    \begin{array}{l}
      \Gamma;\mathcal{H}\vdash_\Sigma a\checks A\\
      \Gamma;\mathcal{H}\vdash_\Sigma b\checks [a/x]B
    \end{array}
  }
  \qquad
  \infer{
    \Gamma;\mathcal{H}\vdash_\Sigma \mathsf{split}([u]C; [v,w]M; p)\infers [p/u]C
  }{
    \begin{array}{l}
      \Gamma;\mathcal{H}\vdash_\Sigma p\infers \Sigma_A[x]B\\
      \Gamma,u:\Sigma_A[x]B;\mathcal{H}\vdash_\Sigma [u]C\checks\uni\\
      \Gamma,v:A,w:[v/x]B;\mathcal{H}\vdash_\Sigma [v,w]M\checks[\langle v,w\rangle/u]C
    \end{array}
  }
  \tag{Sigma}
\end{equation}

%% Should we give Martin-Löf's original induction principle for dependent products, and derive _@_ thence? It would be more symmetrical, though I do not know the degree to which this is a thing people do anymore.

\begin{equation}
  \infer{
    \Gamma;\mathcal{H}\vdash_\Sigma\lambda [x]M\checks\Pi_A[x]B
  }{
    \Gamma,x:A;\mathcal{H}\vdash_\Sigma [x]M\checks[x]B
  }
  \qquad
  \infer{
    \Gamma;\mathcal{H}\vdash_\Sigma f\mathop{@}c \infers [c/x]B
  }{
    \begin{array}{l}
      \Gamma;\mathcal{H}\vdash_\Sigma f\infers\Pi_A[x]B\\
      \Gamma;\mathcal{H}\vdash_\Sigma c\checks A
    \end{array}
  }
  \tag{Pi}
\end{equation}

\end{document}
